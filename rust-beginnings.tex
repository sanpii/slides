% {{{ Packages

% {{{ Versions
% version écran
\documentclass{beamer}

% version orateur
%\setbeameroption{show only notes}
%\usepackage{pgfpages}
%\pgfpagesuselayout{2 on 1}[a4paper,border shrink=5mm]

% version spectateur
%\documentclass[handout]{beamer}
%\usepackage{handoutWithNotes}
%\pgfpagesuselayout{1 on 1 with notes}[a4paper,border shrink=5mm]
% }}}

\usepackage[utf8]{inputenc}
\usepackage[french]{babel}
\usepackage{tikz}
\usepackage{caption}
\usepackage{minted}
\usetheme[numbering=none]{metropolis}

% }}}
% {{{ Commands
\makeatletter
\def\ft@overlay{}

\addtobeamertemplate{footline}{}
{
    \lineskiplimit0pt
    \begin{tikzpicture}[remember picture,overlay]
        \ft@overlay
    \end{tikzpicture}
    \gdef\ft@overlay{}
}

\newcommand<>{\addtooverlay}[1]{
    \only#2{
      \expandafter\gdef\expandafter\ft@overlay\expandafter{\ft@overlay
          \draw[fill=black,opacity=0.70]
              (current page.north east) rectangle (current page.south west);
          \node[text=white] at (current page.center) {
              #1
          };
      }
    }
    \pause{}
}

\newcommand<>{\imgoverlay}[1]{
    \addtooverlay<. (1)>{
        \includegraphics[width=0.9\textwidth,height=0.9\textheight,keepaspectratio]{#1}
    }
}
% }}}

\title{Apprendre un nouveau langage}
\subtitle{Mes débuts avec Rust}
\author{Sanpi}
\date{13 juin 2017}

\begin{document}
% {{{ Titre
\begin{frame}
    \titlepage{}
\end{frame}
\note {
    J’ai commencé à m’intéressé à rust il y a un an, lorsqu’entre deux
    discussions sur Coq, une personne en qui j’ai une grande confiance sur les
    sujets technique m’a dit : « j’utilise deux langages : haskell, et quand je
    ne peux pas faire de fonctionnel, rust ».

    J’avais déjà va passé ce « nouveau » langage entre deux billets sur go et je
    les avais classé dans la même catégorie : des remplaçant du C sans grand
    intérêts.

    Suite à cette discussion, je me suis motivé à regarder rust de plus prêt, ce
    qui fut une brillante idée.
}
% }}}

% {{{
\begin{frame}
    \begin{figure}
        \centering

        \includegraphics[width=70px]{images/lang/php}
        \pause
        \hspace{1in}
        \includegraphics[width=70px]{images/lang/bash}
        \pause
        \hspace{1in}
        \includegraphics[width=70px]{images/lang/javascript}
        \pause
        \hspace{1in}
        \includegraphics[width=70px]{images/lang/rust}
    \end{figure}
\end{frame}
% }}}

% {{{ Suivre un tutoriel
\section{Suivre un tutoriel}
\begin{frame}
    \frametitle{Suivre un tutoriel}
\end{frame}
% }}}

% {{{ Manque d’idée
\section{Manque d’idée}
\begin{frame}
    \frametitle{Manque d’idée}

    \imgoverlay{images/24days}
    \imgoverlay{images/24days-hl}
    \imgoverlay{images/euler}
    \imgoverlay{images/fibonacci}

    \inputminted[
        baselinestretch=1,
        fontsize=\tiny
    ]{rust}{sources/fibonacci.rs}
\end{frame}

\note {
    Mon premier problème à été de trouver quoi faire. À la fois quelque chose de
    simple et d’intéllectuellement satisfaisant, sinon j’allais rapidement
    passer à autre chose.

    C’est en cherchant de la doc sur rust que je suis tombé sur 24 days of rust,
    que je vous conseille de lire, qui (à l’époque) présentait 24 cas d’usage de
    rust dans la vraie vie.

    Et dés le débuts, l’auteur précise que lorsqu’il débute avec un langage, il
    aime bien résoudre quelque problème du projet euleur.

    Alors le projet euler ce sont des problèmes de math, que vous pouver
    résoudre comme bon vous semble, sauf peut être à la main.

    Par exemple le second problème : « calculer la somme des nombres pair de la
    suite de fibonacci jusqu’à 4 millions ».

    Comme je débutais avec rust, j’ai cherché « rust + fibonacci » et je suis
    tombé sur le code d’un itérateur que voici.

    Et c’est en voyant le résultat que j’ai été ému par son élégance.
}

% {{{ Récriture de projets
\section{Récriture de projets}
\begin{frame}
    \frametitle{Récriture de projets}

    \imgoverlay{images/teleinfo}
\end{frame}

\note {
    Ensuite, j’ai cherché des programmes simples à récrire. Ces dernières années
    j’avais tendance à utiliser PHP pour tout et n’importe quoi. Par exemple
    lire les données de consommation électrique via le port série sur un
    raspberry pi.

    Ensuite, pour rester dans le cadre de la domotique, j’ai récri un programme
    écrit en C qui récupère les informations d’une station météo.
}
% }}}

% {{{ Premier crate
\section{Premier crate}
\begin{frame}
    \frametitle{Premier crate}

    \imgoverlay{images/redpitaya}
    \imgoverlay{images/yellow-pitaya}
\end{frame}

\note {
    Pour finir avec mes débuts avec rust, j’ai créé mon premier crate qui est un
    binding pour une bibliothèque C pour piloter un redpitaya qui est un
    oscilloscope/générateur de signal sans écran.

    À l’époque, je l’ai écrit à la main. Ne faite pas ça, il existe bindgen qui
    le fait automatiquement pour vous. Ensuite il vous reste toutefois à écrire
    une surcouche sûr et élégante.

    À l’époque j’avais écris ça pour apprendre, sans forcement d’idée de
    programme à faire avec. Puis, il y a maintenant quelque semaine le suis
    parti pour récrire une bonne partie des logiciels en rust.

    J’ai commencé avec l’interface, car l’application web demande une
    utilisation massive de la sourie et comme le serveur SCPi (qui permet de
    piloter le redpitaya via le réseau) est buggué, j’ai commencé à le récrire
    également en rust.
}
% }}}

% {{{ Quelques conseils
\section{Quelques conseils}
\begin{frame}
    \frametitle{Quelques conseils}

    \imgoverlay{images/rust-fr}
    \imgoverlay{images/awesome}
\end{frame}

\note {
    * Le compilateur est votre ami
    * Comprennez le principe de propriété
    * Un rust awesome avec une énorme liste de chose liée à rust
    * IRC #rust-fr
    * Les meetup rust
}
%}}}

% {{{ Références
\section{Références}
\begin{frame}
    \frametitle{Références}

    \begin{itemize}
        \item \tiny \href{https://github.com/sanpii/slides/releases/download/meetup-rust-2017-05/rust-beginnings.pdf}
            {github.com/\textbf{\alert{sanpii/slides}}/releases/download/meetup-rust-2017-05/rust-beginnings.pdf}
        \item \tiny \url{https://zsiciarz.github.io/24daysofrust/}
        \item \tiny \url{https://projecteuler.net}
        \item \tiny \url{https://hackaday.io/project/12708-uncontrolled-mechanical-ventilation}
        \item \tiny \url{https://crates.io/crates/redpitaya}
        \item \tiny \url{https://servo.github.io/rust-bindgen/}
        \item \tiny \url{https://hackaday.io/project/21383-yellow-pitaya}
        \item \tiny \url{https://blog.guillaume-gomez.fr/Rust/}
        \item \tiny \url{https://github.com/kud1ing/awesome-rust}
    \end{itemize}
\end{frame}
% }}}
\end{document}
