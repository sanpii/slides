% {{{ Packages

% {{{ Versions
% version écran
\documentclass{beamer}

% version orateur
%\setbeameroption{show only notes}
%\usepackage{pgfpages}
%\pgfpagesuselayout{2 on 1}[a4paper,border shrink=5mm]

% version spectateur
%\documentclass[handout]{beamer}
%\usepackage{handoutWithNotes}
%\pgfpagesuselayout{1 on 1 with notes}[a4paper,border shrink=5mm]
% }}}

\usepackage[utf8]{inputenc}
\usepackage[french]{babel}
\usepackage{tikz}
\usepackage{caption}
\usepackage{minted}
\usetheme[numbering=none]{metropolis}

% }}}
% {{{ Commands
\makeatletter
\def\ft@overlay{}

\addtobeamertemplate{footline}{}
{
    \lineskiplimit0pt
    \begin{tikzpicture}[remember picture,overlay]
        \ft@overlay
    \end{tikzpicture}
    \gdef\ft@overlay{}
}

\newcommand<>{\addtooverlay}[1]{
    \only#2{
      \expandafter\gdef\expandafter\ft@overlay\expandafter{\ft@overlay
          \draw[fill=black,opacity=0.70]
              (current page.north east) rectangle (current page.south west);
          \node[text=white] at (current page.center) {
              #1
          };
      }
    }
    \pause{}
}

\newcommand<>{\imgoverlay}[1]{
    \addtooverlay<. (1)>{
        \includegraphics[width=0.9\textwidth,height=0.9\textheight,keepaspectratio]{#1}
    }
}
% }}}

\title{Comment contribuer à Symfony ?}
\author{Nicolas Joseph}
\date{27 avril 2017}

\begin{document}
% {{{ Titre
\begin{frame}
    \titlepage{}
\end{frame}
\note {
    Alors je vais vous parler de comment contribuer à Symfony.
}
% }}}

% {{{ Comment j’ai contribuer à Symfony
\begin{frame}
    \title{Comment \alert{j’ai} contribué à Symfony ?}

    \titlepage{}
\end{frame}

\note {
    Enfin, je vais vous raconter comment j’ai contribué à Symfony en espérant
    vous donner l’envie de faire de même car je suis convaincu que vous avez de
    bonnes idées pour améliorer ce framework.
}

\begin{frame}
    \frametitle{Comment j’ai contribué à Symfony ?}

    \imgoverlay{images/pr}
\end{frame}

\note {
    Je vais donc revenir sur les quatres contributions que j’ai faites il y a
    deux semaines.
}
% }}}

% {{{ Par où commencer
\section{Par où commencer ?}

\begin{frame}
    \frametitle{Par où commencer ?}
\end{frame}

\note {
    Le plus compliqué dans l’histoire est probablement de trouver quelque chose
    à faire. Il existe plein de manière de contribuer à Symfony.
    Étant développeur, je vais me concentrer sur le code.

    Le code de symfony est de très bonne qualité donc il est peux
    probable de tomber sur un bug ou une nouvelle fonctionnalité qui soit
    suffisament simple à implémenter, pour commencer.
}

\begin{frame}[fragile]
    \frametitle{Par où commencer ?}

    \imgoverlay{images/pomm-bundle-pr}
    \imgoverlay{images/symfony4-best-practices}
    \imgoverlay{images/symbiose}
    \imgoverlay{images/knp-dotenv}

    \begin{minted}{php}
<?php
# vlucas/phpdotenv
$dotenv = new Dotenv\Dotenv(__DIR__.'/../');
$dotenv->load();
    \end{minted}

    \begin{onlyenv}<6>
        \begin{minted}{php}
<?php
# symfony/dotenv
$dotenv = new Dotenv\Dotenv();
$dotenv->load(__DIR_ . '/../.env'/*, ... */);
        \end{minted}
    \end{onlyenv}

\end{frame}

\note {
    Ma première idée a été de proposer un moyen de noter un nœud de
    configuration comme obsolète, tout simplement car je besoin c’est imposé au
    détour d’une PR sur le bundle symfony pour pomm.

    J’ai commencé a codé sans réussir à faire quelque chose qui fonctionne. J’ai
    donc abandonné l’idée. Voici pour ma première contribution ratée à symfony.

    Ensuite, au détour des premiers articles de Fabien Potencier sur ce que sera
    symfony 4 et les bonnes pratiques qui vont avec, j’ai voulu mettre à jour
    mon squelette de projet symfony.

    C’est un projet que j’ai commencé afin de rapidement démarré un projet
    incluant symfony, pomm, atoum et behat. Puis continué à l’enrichir au
    fil de des articles et autres conférence sur les bonnes pratiques comme par
    exemple remplacer `app.php` et `app\_dev.php` par un unique `index.php` ou
    encore ajouté un Makefile. Et donc dernière idée en date, était utiliser le
    composant Dotenv de symfony pour charger la configuration depuis des
    variables d’environement plutôt que depuis le fichier parameters.yml. Ce
    sera le cas dans symfony4.

    Je recherche donc à quel endroit il faut faire cela, et je tombe sur un
    article de knp qui propose d’ajouter deux lignes dans le fichier index.php.
    Lignes que je copie/colle et là rien ne se passe…

    J’avais simplement oublié de lire le début de l’article qui précise qu’il
    s’agit du composant dotenv de vlucas, pas celui de symfony qui s’utilise
    comme ceci.

    Donc plutôt que de passer un dossier au constructeur, on passe un fichier à
    la méthode load.
}
% }}}

% {{{ Coder
\section{Coder}

\begin{frame}
    \frametitle{Coder}

    \imgoverlay{images/dotenv-pr}
\end{frame}

\note {
    Donc une fois que vous avez votre sujet de contribution, il reste à la
    codée.

    J’ai donc proposé deux PR, une première qui vérifie que l’on charge bien un
    fichier et une seconde pour vérifier qu’il y a au moins un argument à la
    méthode load.

    J’ai fait deux PR car je n’étais pas sûr que la seconde soit acceptée.

    Donc voici ma première PR, neuf lignes de test et une demie-ligne de code.
}
% }}}

% {{{ Créer la PR
\section{Créer la PR}

\begin{frame}
    \frametitle{Créer la PR}

    \imgoverlay{images/create-pr}
    \imgoverlay{images/contributing}
    \imgoverlay{images/checklist}
\end{frame}

\note {
    Bon, nous avons notre PR, c’est maintenant que les choses sérieuses vont
    commencer.

    En créant votre PR sur github, vous serez invité à un peu de lecture que je
    vous conseille de lire avant de commencer à coder.

    Ensuite, vous devez remplir une check list.

    Une fois fait, on valide la création de la PR et c’est là que les robots va
    entrée en jeux.
}
% }}}

% {{{ Les robots
\section{Les robots}

\begin{frame}
    \frametitle{Les robots}

    \imgoverlay{images/carsonbot}
    \imgoverlay{images/fabbot}
    \imgoverlay{images/checks}
\end{frame}

\note {
    Le premier robot qui va intervenir, c’est carsonbot, dont le code est libre,
    et va s’occuper de mettre les labels adécouate en fonction du titre de votre
    PR. Manque de bol ici, dotenv n’est pas encore pris en compte. Ainsi que le
    label « Needs Review ». Je reviendrais dessus par la suite.

    Le second robot, fabbot, va faire quelques vérifications rapides.

    Ensuite, on trouve des boots plus classique, comme travis qui va lancer les
    tests unitaire et AppVeyor qui fait la même chose mais sous Windows.

    Donc si tout cela est ok, vous avez fini la première partie de votre
    travail.
}
% }}}

% {{{ Les humains
\section{Les humains}

\begin{frame}
    \frametitle{Les humains}

    \imgoverlay{images/reviewed}
    \imgoverlay{images/memory}
    \imgoverlay{images/fabpot}
\end{frame}

\note {
    C’est à ce moment qu’intervient les humains. C’est un peu plus chaotique,
    mais globalement on peux les classer dans 3 catégories.

    La première catégorie, ce sont les personnes qui vont
    relire votre contribution. Chacun peut le faire et laisser un commentaire
    pour mettre à jour le status de la PR.

    Ensuite, des gens qui on un peu suivi les choses vont vous indiquer que ce
    que vous proposé à déjà été refusé.

    Bon et bien sûr des gens qui vont être d’accord ou pas d’accord avec votre
    idée. Bref, du classique si vous avez déjà participé à un projet.

    Peut être une chose intéressante à noter c’est que Fabien Potencier semble
    avec le dernier mot, en mergeant votre PR ou en la fermant.
}
% }}}

% {{{ Conclusion
\section{Conclusion}

\begin{frame}
    \frametitle{Conclusion}
\end{frame}

\begin{frame}
    \frametitle{Conclusion}

    \imgoverlay{images/pr}
\end{frame}

\note {
    Voilà, mon code est disponible pour tout les utilisateurs de symfony et j’ai
    à mon niveau amélioré cet exellent outils.

    En guise de conclusion, je vous conseilleai de commencer avec une petite
    contribution, de préférence un bug qui aura le plus de change d’être
    intégré.
}
% }}}

% {{{ Références
\section{Références}
\begin{frame}
    \frametitle{Références}

    \begin{itemize}
        \item \tiny \href{https://github.com/sanpii/slides/releases/download/sfpot-2017-04/symfony-contrib.pdf}
            {github.com/\textbf{\alert{sanpii/slides}}/releases/download/sfpot-2017-04/symfony-contrib.pdf}

        \item \tiny \url{https://github.com/symfony/symfony/blob/master/CONTRIBUTING.md}
        \item \tiny \url{http://fabien.potencier.org/symfony4-best-practices.html}
        \item \tiny \url{https://knpuniversity.com/screencast/micro-symfony/dotenv}
        \item \tiny \url{https://github.com/carsonbot/carsonbot}
        \item \tiny \url{http://fabbot.io}
    \end{itemize}
\end{frame}
% }}}
\end{document}
